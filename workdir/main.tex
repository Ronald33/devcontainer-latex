% Templates:
%   https://bibtex.eu/es/
%   https://bibtex.eu/es/overleaf/#latex-basics

\documentclass{article}

\usepackage[utf8]{inputenc}
\usepackage{graphicx}
\usepackage{caption}
\usepackage{float}

\graphicspath{ {./images/} }

\title{My First LaTeX Document}
\author{Your Name}

\begin{document}

\maketitle

\section{Introduction}
This is a sample document demonstrating the basics of LaTeX.

\section{Formatting}
You can format text in \textbf{bold}, \textit{italic}, \underline{underline}, or \texttt{typewriter} font.

\section{Lists}
Here's an example of a bulleted list:
\begin{itemize}
    \item First item
    \item Second item
    \item Third item
\end{itemize}

\section{Mathematics}
LaTeX is great for typesetting mathematical formulas. Here's an example of an equation:
\begin{equation}
    E = mc^2
\end{equation}

\section{Figures and Tables}
You can include figures and tables in your document. Here's an example of a figure:
\begin{figure}[H]
    \centering
    \includegraphics[width=0.5\textwidth]{example-image}
    \caption{An example figure}
    \label{fig:example}
\end{figure}

And here's an example of a table:
\begin{table}[H]
    \centering
    \begin{tabular}{|c|c|}
        \hline
        \textbf{Item} & \textbf{Quantity} \\
        \hline
        Apple & 3 \\
        Orange & 5 \\
        \hline
    \end{tabular}
    \caption{An example table}
    \label{tab:example}
\end{table}

\section{References}
You can refer to labeled sections, equations, figures, and tables. For example, see Figure~\ref{fig:example} and Table~\ref{tab:example}.

\section{Citations and Bibliography}
Cite references using \cite{Hemingway1952}. Here's an example citation.

\bibliographystyle{unsrt}
\bibliography{references}

\end{document}